\documentclass[12pt]{article}

\usepackage{cmap}

\usepackage[utf8]{inputenc}
\usepackage[T1]{fontenc}


\begin{document}
\title{Pretraživanje prostora stanja}

\section{Pretraživanje prostora stanja}

\subsection{Opis problema}

Neformalno rečeno, krenuvši iz početnog stanja želimo pronaći put do konačnog.

Sa $S$ označimo skup svih stanja (to još nazivamo i prostor stanja). Neka nam
$s_0 \in S$ označava početno stanje. Skup konačnih stanja (ciljeve) nećemo
definirati eksplicitno, nego preko funkcije $goal : S \rightarrow \lbrace \top,
\bot \rbrace$. Kako bismo znali u koja sve stanja možemo prijeći iz nekog
stanja, definirat ćemo i funkciju prijelaza (sljedbenika) kao $succ : S
\rightarrow \wp(S)$, s time da skup sljedbenika ne mora biti eksplicitno
vraćen, nego može biti i iterator. Dodatno

Problem pretraživanja stanja svede se na problem pretraživanja \textbf{
usmjerenog grafa} (digrafa - directed graph). Vrhovi grafa nam predstavljaju
stanja, a lukovi (bridovi) prijelaze. Ako prijelazi imaju cijene, koristit ćemo
\textbf{usmjeren težinski graf}. \textbf{Stablo pretraživanja} gradimo tako da
se, krenuvši iz prvog čvora, nekom \textbf{strategijom pretraživanja} proširimo
u sve njemu susjedne. \textbf{Otvorenim čvorovima (frontom)} nazivamo one koje
smo generirali, ali još nismo proširili, dok su \textbf{zatvoreni čvorovi} oni
koji su već prošireni. Kako ćemo se proširiti u sva susjedna stanja, bez obzira
jesmo li stanja već posjetili, stablo pretraživanje bit će beskonačno ako graf
sadrži cikluse. 

\textbf{Čvor} ćemo definirati kao podatkovnu strukturu koja pohranjuje stanje
i još neke podatke (npr. dubinu). 


\end{document}
